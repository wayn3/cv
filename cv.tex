\documentclass[a4paper, 11pt, DIV=15,headings=normal]{scrartcl}
\linespread{1.15}
\usepackage[automark,headsepline]{scrpage2}
%\usepackage[T1]{fontenc}
%\usepackage{titlesec}
\usepackage[%
    backend=biber,maxbibnames=8,%
    isbn=false,sorting=ydnt,%
    style=phys,
    biblabel=brackets,pageranges=false]{biblatex}
\renewcommand*{\bibfont}{\footnotesize}
\usepackage[colorlinks, breaklinks, urlcolor=blue]{hyperref}
\usepackage{tgtermes}
\usepackage{mhchem}

%\renewcommand{\familydefault}{\sfdefault}
%\usepackage{sfmath}

%\addsectionbib{cv.bib}

\addbibresource{cv.bib}

\defbibenvironment{bibselect}
  {\itemize}
  {\enditemize}
  {\item}

\defbibheading{bibselect}[\refname]{%
    \section*{#1}}

\begin{document}

\begin{minipage}{0.50\linewidth}
{\Large {Wei \textsc{Chen}}}\\
%Nationality: Chinese \\
%Date of Birth: 11 August 1980 \\
Place of Birth: Shanghai, China \\
Present residence: Wavre, Belgium \\
\small \url{https://orcid.org/0000-0002-7496-0341} \\
\small \url{https://github.com/wch3n}
\end{minipage}\hfill
\begin{minipage}{0.45\linewidth}
Universit\'{e} Catholique de Louvain\\
Institute of Condensed Matter and Nanosciences\\
Chemin des \'{E}toiles, 8 bte L7.03.01 \\
B-1348, Louvain-la-Neuve, Belgium \\
\texttt{wei.chen\_at\_uclouvain.be}
\end{minipage}
\vspace{6mm}

\section*{Education}
\begin{tabular}{ll}
08.2007 -- 04.2011 & Dr.\ rer.\ nat, \emph{summa cum laude}  \\
                   & Leibniz Universit\"{a}t Hannover, Germany\\
                   & Institut f\"{u}r Festk\"{o}rperphysik (Prof. Herbert Pfn\"{u}r) \\
                   & $\bullet$ Adsorption of organic molecules on wide band-gap insulators \\
07.2005 -- 08.2007 & Research assistant, Fudan University\\
                   & Institute of Microelectronic \\
                   & $\bullet$ High-$\mathbf{k}$ dielectrics in nonvolatile memory \\
09.2002 -- 07.2005 & MSc.\ in Electrical Engineering, Fudan University \\
                   & Institute of Microelectronics  \\
                   & $\bullet$ DFT study on atomic layer deposition of high-$\mathbf{k}$ gate dielectrics\\
09.1998 -- 07.2002 & BSc., Fudan University, Shanghai, China
\end{tabular}

\section*{Research Experience}
\begin{tabular}{ll}
06.2016 -- present & Post-doctoral researcher, UCLouvain, Belgium \\
                   & Institute of Condensed Matter and Nanosciences\\ 
                   & Profs. Gian-Marco Rignanese and Geoffroy Hautier  \\
                   & $\bullet$ Many-body perturbation theory \\
                   & $\bullet$ Nonempirical hybrid density functional \\
                   & $\bullet$ High-throughput computational screening \\
                   & $\bullet$ High-entropy alloys \\
                   & $\bullet$ Thin-film solar cells \\
                
04.2011 -- 03.2016 & Post-doctoral researcher, Ecole Polytechnique F\'{e}d\'{e}rale de Lausanne (EPFL) \\
                   & Chair of Atomic Scale Simulation  \\
                   & Prof. Alfredo Pasquarello \\
                   & $\bullet$ Defects in semiconductors and insulators \\
                   & $\bullet$ Interfaces in semiconductor heterojunctions  \\
                   & $\bullet$ Advanced electronic-structure methods: $GW$ approximation and hybrid functionals \\
                   & $\bullet$ Electronic structure of liquid water: many-body and nuclear quantum effects
\end{tabular}

\section*{Coding skills}
\begin{tabular}{l}
Good knowledge of \textsc{Fortran}, \textsc{C}, \textsc{Python}, and \textsc{gnu bash}.\\
Well versed in parallel computing (\textsc{mpi}). \\
Active developer for \href{http://www.abinit.org}{\textsc{abinit}} and 
\href{http://www.quantum-espresso.org}{\textsc{Quantum espresso}}. \\
\end{tabular}


\section*{Code developments}
\begin{tabular}{ll}
\textsc{abinit}  & Bootstrap exchange-correlation kernel for accurate $GW$ quasiparticle energies, \\
                 & Gygi-Baldereschi auxiliary function for the treatment of
Coulomb singularity \\
\textsc{quantum-espresso} & Range-separated hybrid density functional \\
\textsc{fnv}     & \textsc{Python} class for finite-size-corrections of
periodic charged defects \\
                 & (\url{https://github.com/wayn3/FNV})
\\
\end{tabular} 

\section*{Teaching Activities}
\begin{tabular}{ll}
03.2019 -- 06.2019 & Masters course: Atomistic and Nanoscopic Simulations \\
                   & Teaching assistance, UCLouvain\\
04.2012 -- 04.2016 & Masters course: Computational Simulation and Physical Systems I \& II \\
                   & Teaching assistance, EPFL\\
06.2013 -- 04.2014 & Supervising a Masters student (Karim Steiner) \\
                   & Project: Band-offset of lattice matched semiconductor heterojunctions 
\end{tabular}

\section*{Recent Talks and Seminars}
\begin{tabular}{ll}
09.2018         & Swiss Physics Society Annual Meeting 2018, Lausanne, Switzerland \\
                & Invited talk: ``Electronic structures through $GW$ and hybrid functionals''\\
08.2017         & 29th International Conference on Defects in Semiconductors (ICDS), Matsue, Japan  \\
                & Invited talk: ``Towards accurate determination of defect levels in semiconductors'' \\
05.2017         & \textsc{abinit} Developer Workshop 2017, Fr\`{e}jus, France \\
                & Invited talk: ``Accurate band gaps via efficient vertex corrections in $GW$''\\
10.2015         & Universit\'{e} catholique de Louvain, Belgium \\
                & Invited seminar: ``Efficient vertex corrections in $GW$'' \\
09.2015         & \textsc{Psi-k} 2015 conference, San Sebastian, Spain \\
                & Talk: ``Accurate band gaps via efficient vertex corrections in $GW$'' \\
07.2015         & International Conference on Defects in Semiconductors (ICDS 15), Espoo, Finland \\
                & Talk: ``Determination of defect energy levels through $GW$''\\
04.2015         & ``Nothing is Perfect`` workshop, Ascona, Switzerland \\
                & Invited Talk: ``First-principles determination of defect energy levels through $GW$'' \\
08.2014         & International conference on the physics of semiconductors (ICPS 14), Austin, USA \\
                & Talk: ``Band offset of lattice-matched semiconductor heterojunctions'' \\
07.2013         & International Conference on Defects in Semiconductors (ICDS 13), Bologna, Italy \\
                & Talk: ``Defect energy levels: Hybrid functionals vs $GW$'' \\
\end{tabular}


\section*{Miscellaneous}
\begin{tabular}{l}
Referee for \textit{Phys.\ Rev.\ Lett.}, \textit{Phys.\ Rev.\ B}, \textit{Appl.\ Phys. Lett.},
\textit{J. Phys. Chem. Lett.}, \textit{J. Phys. Chem.}
\end{tabular}

\newrefsegment
\AtNextBibliography{\small}
\nocite{Chen2019,Chen2016,Chen2015,Chen2017,Chen2018a}
\printbibliography[segment=1,heading=bibselect,title={Selected Publications},env=bibselect]

\newrefsegment
\AtNextBibliography{\small}
\nocite{*}
\printbibliography[resetnumbers=true,title={Publications in Chronological Order}]

\bigskip

\begin{minipage}{0.4\linewidth}
\scriptsize Updated on \today.
\end{minipage}

\end{document}
